\chapter{群的基本概念}
\section{群的相关概念}
\begin{definition}[群]
    集合$G$中有二元运算,并且满足如下四条群公理,则称为群:
    \begin{enumerate}
        \item 封闭性:$\forall R,S\in G,\quad RS\in G$
        \item 结合律:$\forall R,S,T\in G,\quad R(ST)=(RS)T$
        \item 恒元:$\exists E\in G,\forall R\in G,\quad ER=R$
        \item 逆元:$\forall R\in G,\exists R^{-1}\in G,\quad R^{-1}R=E$
    \end{enumerate}
\end{theorem}
在群的定义中,对恒元和逆元只要求左乘成立,但其实从此定义出发可以证明,上述性质在右乘时也成立:
\begin{property}[恒元和逆元的右乘]
    \indent
    \begin{itemize}
        \item $ER=RE=R$
        \item $R^{-1}R=RR^{-1}=E$
    \end{itemize}
\end{property}
由此出发可以证明群中恒元和逆元的唯一性:
\begin{property}[恒元和逆元的唯一性]
    \indent
    \begin{itemize}
        \item 恒元唯一性:若$TR=R$,则$T=E$
        \item 逆元唯一性:若$TR=E$,则$T=R^{-1}$
    \end{itemize}
\end{property}
关于幂运算,还有如下性质
\begin{property}[幂运算]
    \indent
    \begin{itemize}
        \item $(RS)^{-1}=S^{-1}R^{-1}$
        \item $R^mR^n=R^{m+n}$
        \item $(R^m)^n=R^{mn}$
    \end{itemize}
\end{property}